The staging of lung cancer is crucial for determining the extent of the disease and guiding the appropriate treatment plan. Lung cancer staging is generally based on the size of the tumor, its location, and whether it has spread to nearby lymph nodes or other parts of the body.

\section{Stages of Non-Small Cell Lung Cancer (NSCLC)}
Non-Small Cell Lung Cancer (NSCLC) is classified into the following stages

\begin{itemize}
    \item \textbf{Stage 0 (In Situ)} The cancer is localized and confined to the inner lining of the lungs. At this stage, the tumor has not invaded deeper tissues or spread to other areas.
    \item \textbf{Stage I} The tumor is small (less than 4 cm) and limited to the lung, without spreading to lymph nodes.
    \item \textbf{Stage II} The tumor may be larger or have spread to nearby lymph nodes or tissues, such as the chest wall or diaphragm.
    \item \textbf{Stage III} The cancer has spread to more distant lymph nodes or other nearby structures, such as the heart or trachea.
    \item \textbf{Stage IV} The cancer has \textcolor{red}{metastasized}, meaning it has spread to distant parts of the body, such as the liver, brain, or bones.
\end{itemize}

\section{Stages of Small Cell Lung Cancer (SCLC)} 
Small Cell Lung Cancer (SCLC) is typically classified into two broad stages due to its aggressive nature:

\begin{itemize}
    \item \textbf{Limited Stage} The cancer is confined to one side of the chest, involving only one lung and nearby lymph nodes.
    \item \textbf{Extensive Stage} The cancer has spread to the other lung, distant lymph nodes, or other parts of the body.
\end{itemize}

\section{TNM Classification} 
The \textcolor{red}{TNM} system is also commonly used to describe lung cancer stages:
\begin{itemize}
    \item \textbf{T (Tumor)}  Refers to the size and extent of the primary tumor.
    \item \textbf{N (Nodes)} Indicates whether cancer has spread to nearby lymph nodes.
    \item \textbf{M (Metastasis)} Describes whether the cancer has spread to distant parts of the body.
\end{itemize}

Understanding the stage of lung cancer helps oncologists design personalized treatment plans; including surgery, \textcolor{red}{chemotherapy}, \textcolor{red}{radiation therapy}, or \textcolor{red}{targeted therapies}. Early-stage detection offers a higher chance of successful treatment, while advanced stages require a more comprehensive approach.
