Management and treatment of lung cancer vary depending on the type, stage, and overall health of the patient. Advances in medical technology and understanding of the disease have greatly improved treatment options, offering patients personalized and targeted approaches.

\section{Surgical Options}
According to \cite{hoy2019surgical}, surgery is often recommended for early-stage non-small cell lung cancer (NSCLC) if the cancer is localized and the patient is healthy enough to undergo the procedure. Common surgical procedures include:
\begin{itemize}
    \item \textbf{Lobectomy:} Removal of the affected lobe of the lung.
    \item \textbf{Pneumonectomy:} Removal of an entire lung in cases of extensive disease.
    \item \textbf{Segmentectomy or Wedge Resection:} Removal of a smaller portion of the lung, typically for patients unable to tolerate more extensive surgery.
\end{itemize}

\section{Radiation Therapy}
According to \cite{de2013state}, Radiation therapy uses high-energy rays to destroy cancer cells. It can be used in conjunction with surgery or as a standalone treatment in cases where surgery isn’t an option. Techniques include:
\begin{itemize}
    \item \textbf{External Beam Radiation Therapy (EBRT):} Targets cancer cells from outside the body.
    \item \textbf{Stereotactic Body Radiation Therapy (SBRT):} A precise and focused form of radiation used for small tumors.
\end{itemize}

\section{Chemotherapy}
According to \cite{ihde1992chemotherapy}, Chemotherapy involves the use of drugs to kill cancer cells or stop their growth. It is commonly used in advanced stages of both NSCLC and small cell lung cancer (SCLC) and may be combined with other treatments. Commonly used drugs include cisplatin and carboplatin.

\section{Targeted Therapy}
According to \cite{mayekar2017current}, Targeted therapy focuses on specific genetic mutations or proteins that drive cancer growth. It is most effective in NSCLC with identifiable mutations, such as:
\begin{itemize}
    \item \textbf{EGFR inhibitors} (e.g., erlotinib, gefitinib)
    \item \textbf{ALK inhibitors} (e.g., crizotinib, alectinib)
    \item \textbf{ROS1 inhibitors}
\end{itemize}

\section{Immunotherapy}
According to \cite{steven2016immunotherapy}, Immunotherapy helps the immune system recognize and attack cancer cells. Drugs like immune checkpoint inhibitors (e.g., pembrolizumab, nivolumab) have shown promise, particularly in advanced NSCLC.

\section{Palliative Care}
Palliative care focuses on improving quality of life by managing symptoms such as pain, breathlessness, and fatigue. It is an essential part of treatment for advanced lung cancer and may include medications, counseling, and support for patients and families, according to \cite{ferrell2011palliative}.

\section{Lifestyle and Supportive Measures}
Patients are encouraged to adopt healthy habits, such as quitting smoking and eating a balanced diet, to improve overall outcomes. Emotional and psychological support, including therapy and support groups, plays a critical role in the treatment journey, according to \cite{heredia2023effectiveness}.

\section{Future Directions}
Emerging therapies, including gene therapy and personalized medicine, hold promise for the future of lung cancer treatment. Ongoing research aims to improve survival rates and quality of life for patients.

An integrated approach combining different therapies tailored to each patient's needs is crucial for effective lung cancer management.