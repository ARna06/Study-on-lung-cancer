\begin{remark}
Early and accurate diagnosis of lung cancer is crucial for effective treatment and management. According to \cite{deng2024prospects} the diagnostic process typically involves a combination of \textcolor{red}{imaging studies} , \textcolor{red}{laboratory tests} , and \textcolor{red}{biopsy procedures} to confirm the presence of cancer and determine its type and stage.
\end{remark}

\section{Imaging Tests} 
Imaging studies are often the first step in diagnosing lung cancer. These tests help visualize abnormalities in the lungs and surrounding tissues.

\begin{highlight}
\begin{itemize}
    \item \textbf{Chest X-ray} Often the initial test used to detect unusual masses or spots in the lungs\cite{kalkeseetharaman2024bird}.
    \item \textbf{Computed Tomography (CT) Scan} Provides detailed cross-sectional images of the lungs and chest, helping to identify the size, shape, and location of tumors\cite{hendrick2024benefit}.
    \item \textbf{Positron Emission Tomography (PET) Scan} Helps determine if the cancer has spread by detecting areas of increased metabolic activity.
    \item \textbf{Magnetic Resonance Imaging (MRI)} Used in specific cases, especially when there is a suspicion of cancer spreading to the brain or spinal cord\cite{li2024application}.
\end{itemize}
\end{highlight}

\section{Laboratory Tests}
Lab tests analyze blood, sputum, or other samples to detect cancer-related abnormalities.

\begin{highlight}
\begin{itemize}
    \item \textbf{Sputum Cytology} \cite{wen2024methylated}Examines mucus produced by coughing for the presence of cancer cells, particularly in central lung tumors.
    \item \textbf{Blood Tests} While not diagnostic for lung cancer, blood tests can help assess overall health and detect markers that may indicate cancer.
\end{itemize}
\end{highlight}

\section{Biopsy Procedures} 
A biopsy involves the removal of tissue or fluid samples for microscopic examination. It is the definitive method for diagnosing lung cancer and determining its type:

\begin{highlight}
\begin{itemize}
    \item \textbf{Bronchoscopy} \cite{rozman2024interventional}A thin, flexible tube is inserted through the nose or mouth to examine the airways and collect tissue samples.
    \item \textbf{Needle Aspiration (Fine Needle Aspiration or Core Biopsy)} \cite{lee2024additional} A needle is used to extract cells or tissue from the lung or lymph nodes, often guided by imaging techniques.
    \item \textbf{Thoracentesis} \cite{dahlberg2024thoracentesis} Removes fluid from around the lungs (pleural effusion) to check for cancer cells.
    \item \textbf{Surgical Biopsy} \cite{mahuron2024applications} Performed when other methods are inconclusive; includes procedures like \textcolor{red}{thoracoscopy} or \textcolor{red}{thoracotomy}.
\end{itemize}
\end{highlight}

\section{Molecular and Genetic Testing}
\begin{outline}
Advances in cancer treatment have led to the use of molecular and genetic testing of tumor samples\cite{kanasaki2024upfront}. These tests identify specific mutations (e.g., \textcolor{red}{EGFR, ALK, KRAS}) and biomarkers that can guide personalized treatment, such as \textcolor{red}{targeted therapies} or \textcolor{red}{immunotherapies}\cite{li2024targeted}.
\end{outline}

\section{Staging Tests} 
Once a lung cancer diagnosis is confirmed, staging tests are conducted to determine the extent of the disease. These may include additional imaging studies like \textcolor{red}{bone scans}, \textcolor{red}{PET-CT}, or \textcolor{red}{MRI}, and lymph node evaluation using \textcolor{red}{mediastinoscopy} or \textcolor{red}{endobronchial ultrasound (EBUS)}\cite{de2024future}.

\begin{remark}
Accurate diagnosis and staging enable oncologists to develop tailored treatment plans, improving outcomes and quality of life for patients.
\end{remark}